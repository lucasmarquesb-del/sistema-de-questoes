% ============================================
% TEMPLATE PADRÃO - SISTEMA DE QUESTÕES
% Versão: 1.0.1
% ============================================

\documentclass[12pt,a4paper]{article}

% Pacotes essenciais
\usepackage[utf8]{inputenc}
\usepackage[brazilian]{babel}
\usepackage[margin=2cm]{geometry}

% Matemática
\usepackage{amsmath,amssymb,amsfonts}
\usepackage{mathtools}

% Imagens
\usepackage{graphicx}
\usepackage{float}

% Listas personalizadas
\usepackage{enumitem}

% Multicolunas (opcional)
\usepackage{multicol}

% Cabeçalho e rodapé
\usepackage{fancyhdr}

% Links e referências
\usepackage{hyperref}
\hypersetup{
    colorlinks=true,
    linkcolor=black,
    urlcolor=blue,
    pdfauthor={Sistema de Questões},
    pdftitle={Lista de Exercícios}
}

% Configurações de página
\pagestyle{fancy}
\fancyhf{}
\rhead{\thepage}
\lhead{Lista de Exercícios}
\renewcommand{\headrulewidth}{0.4pt}

% Espaçamento
\setlength{\parindent}{0pt}
\setlength{\parskip}{6pt}

% ============================================
% INÍCIO DO DOCUMENTO
% ============================================

\begin{document}

% CABEÇALHO PERSONALIZADO
\begin{center}
    {\Large\textbf{[TITULO_LISTA]}}
\end{center}

\vspace{0.5cm}

\noindent
\textbf{[NOME_ESCOLA]} \\
Professor: [NOME_PROFESSOR] \\
Data: [DATA] \hfill Turma: [TURMA]

\vspace{0.5cm}

% INSTRUÇÕES
\noindent
\textit{[INSTRUCOES]}

\vspace{1cm}

% ============================================
% QUESTÕES
% ============================================

\begin{enumerate}[leftmargin=*]

% [QUESTOES_AQUI]
% Formato para cada questão:
%
% \item \textbf{([ANO] - [FONTE])}
%
% [ENUNCIADO]
%
% \begin{center}
%     \includegraphics[scale=0.7]{[CAMINHO_IMAGEM]}
% \end{center}
%
% \begin{enumerate}[label=\Alph*)]
%     \item [ALTERNATIVA A]
%     \item [ALTERNATIVA B]
%     \item [ALTERNATIVA C]
%     \item [ALTERNATIVA D]
%     \item [ALTERNATIVA E]
% \end{enumerate}
%
% \vspace{[ESPACO_RESPOSTA]cm}

\end{enumerate}

% ============================================
% GABARITO (opcional)
% ============================================

\newpage

\section*{Gabarito}

\begin{enumerate}[leftmargin=*]
% [GABARITO_AQUI]
% Formato: \item Questão 1: [LETRA_CORRETA]
\end{enumerate}

% ============================================
% RESOLUÇÕES (opcional)
% ============================================

\newpage

\section*{Resoluções}

\begin{enumerate}[leftmargin=*]
% [RESOLUCOES_AQUI]
% Formato:
% \item \textbf{Questão 1:}
%
% [RESOLUCAO_LATEX]
\end{enumerate}

\end{document}
